\documentclass{article}

\usepackage[utf8]{inputenc}
\usepackage[T1]{fontenc}
\usepackage[english]{babel}
\usepackage[left=3cm, right=3cm, bottom=2cm, top=2cm]{geometry}
\usepackage{fixltx2e}
\usepackage{fancyhdr}
\usepackage{graphicx}
\usepackage{amsmath}
\usepackage{amssymb}
\usepackage{mathrsfs}
\usepackage{qtree}
\usepackage{tipa}
\usepackage{indentfirst}
\usepackage[shortlabels]{enumitem}
\usepackage{stmaryrd}
\usepackage{qtree}
\usepackage{hyperref}

\MakeRobust{\overrightarrow}

\pagestyle{fancy}
\lhead{Compte rendu}
\rhead{Luc Chabassier \and Tomas Rigaux}

\newcommand{\vc}[1]{\overrightarrow{#1}}
\newcommand{\mirrorh}[1]{\raisebox{\depth}{\scalebox{1}[-1]{#1}}}
\newcommand{\sss}{\subset}
\renewcommand{\ss}{\subseteq}
\newcommand{\rsss}{\supset}
\newcommand{\rss}{\supseteq}
\newcommand{\lbr}{\llbracket}
\newcommand{\rbr}{\rrbracket}
\qtreecenterfalse

\renewcommand{\P}{\mathbb{P}}
\newcommand{\Pf}{\mathbb{P}_{fin}}
\newcommand{\rD}{\rightarrow_D}
\newcommand{\sem}[1]{\lbr #1 \rbr}
\newcommand{\dsem}[1]{D\sem{#1}}
\newcommand{\Ec}{\mathcal{E}_{\mathcal{C}}}
\newcommand{\esem}[1]{\Ec\sem{#1}}
\newcommand{\rE}{\rightarrow_{\Ec}}
\renewcommand{\vec}[1]{\overrightarrow{#1}}

\title{TCP/IP pour Hurd}
\author{Luc Chabassier \and Tomas Rigaux}

\begin{document}
\maketitle

\section{Introduction}
Hurd est un kernel de système d'exploitation, ayant initialement pour ambition
d'être le noyau du projet GNU. Des difficultés techniques lors de son développement
ont conduit à son abandon en faveur de Linux. Malgré tout, quelques irréductibles
développent encore et toujours ce système.

Malgré son abandon, Hurd se base sur des concepts intéressants, inspirés de Plan 9
et partiellements repris dans Linux avec FUSE. L'idée de base est de déplacer
le plus possible de services et de driver en user-space. Il suffit donc
d'avoir un micro-noyau pour faire tourner l'ensemble des services. Celui utilisé
par Hurd est la réimplémentation par le projet GNU de Mach.

Dans le cas qui nous intéresse, la couche TCP/IP de Hurd actuelle (nommée pfinet,
qui est un hack sur la TCP/IP de Linux datant d'il y a plusieurs années) est un
service comme un autre, placé en user space. Notre objectif est de fournir une
autre implémentation, plus proche de la philosophie de Hurd, c'est à dire non
monolithique mais découpée en plusieurs services, chacun implémentant un
protocole.

Rétrospectivement, la partie qui nous a le plus causé problème est la compréhension
de l'API Mach et de l'écosystème de Hurd, qui souffre d'une absence de documentation
presque absolue.

\section{Le micronoyau Mach}

Comme tout micronoyau qui se respecte, Mach se concentre sur la gestion des processus
et leur communication.

Le système d'IPC est centré sur la notion de port : un port est une queue de messages
typés gérée par le kernel. Les processus ne manipulent pas la queue directement mais
des droits sur le port. Les droits sur le port sont de \emph{capabilities},
que les processus peuvent transférer ou utiliser pour recevoir ou envoyer des
messages sur un port donné.

Se pose alors un problème particulier : comment faire communiquer deux processus
distincts qui n'ont pas de lien de parenté. L'un doit créer un port et envoyer
un droit à l'autre. Mais pour envoyer un droit à l'autre processus, il doit
déjà exister un port permettant de les faire communiquer : la nécessité d'un
tier initiant la communication est nécessaire. Mach parle d'un serveur de noms.
C'est ici que Hurd intervient, avec une abstraction classique : le système de
fichiers. Au lieu d'avoir un système de fichier, chaque fichier est un port implémentant
un protocole particulier lui permettant de se comporter comme un fichier.
Chaque processus est lancé avec connaissance de la racine et de son dossier courant,
et il peut s'en servir pour négotier des ports. Cette idée est très proche de
ce que faisait plan 9, ou ce qui se passe avec \texttt{/proc} et \texttt{/dev}
sous Linux.

Un processus qui implémente l'interface de fichier est appelé un translator.
L'intérêt est qu'il peut notament implémenter d'autres interfaces en même temps.
Les drivers sont donc des fichiers qui implémentent en plus l'interface \emph{device},
\emph{pfinet} implémente en plus l'interface \emph{socket} \ldots

\section{Notre projet}

L'objectif est d'implémenter une couche TCP/IP minimaliste sous la forme de
4 translators :\begin{itemize}
    \item un translator gérant le protocole \emph{Ethernet}, gérant l'interface
        avec le \emph{device} gérant ethernet, et répartissant les packets
        reçus à différents serveurs de niveau supérieur en fonction de leur
        ethertype.
    \item un translator gérant le protocole \emph{ARP}, répondant aux requêtes
        d'ethertype \texttt{0806} et fournissant une interface permettant aux
        autres translator de faire des requêtes pour obtenir des addresses de
        niveau 2 correspondant.
    \item un translator gérant le protocol \emph{IP}, qui à nouveau répartit
        les frames entrantes en fonction de leur type.
    \item un translator gérant le protocol \emph{TCP}, implémentant à la fois
        une interface pour discuter avec le translator \emph{IP} et l'interface
        \emph{socket} pour offrir une compatibilité avec \texttt{pfinet}.
\end{itemize}

\section{Ce qui a été fait}

\subsection{Translator ethernet}
Le translator ethernet se connecte sur un device Mach, et écoute sur son fichier
des ethertypes en ASCII. Lorsqu'un éthertype est écrit sur son fichier, il suppose
que c'est un enregistrement d'un \emph{handler} pour cet ethertype. Il va donc
ouvrir le fichier correspondant (dans un dossier choisit à la compilation, ayant
pour nom l'ethertype), et y envoyer un message contenant un port de réponse,
l'adresse MAC et la taille d'une addresse MAC. Puis il y transmet le contenu
de chaque frame ayant le bon ethertype. De plus, tous les paquets envoyés sur
le port de réponse son encapsulés dans une frame ethernet avant d'être transmis
sur le réseau.

Il supporte uniquement les frames Ethernet II et Novell IPX : il ne supporte pas
les extensions SNAP ou LLC.

\subsection{Translator ARP}
Le translator ARP implémente l'interface d'un \emph{handler} pour l'éthertype
\texttt{0x0806}, capable de gérer le protocole ARP dans sa forme la plus générale :
il peut s'adapter (non testé cependant) à d'autres types d'adresses réseau, et
supporte simultanément plusieurs types d'adresses ethernet. Plus précisément,
quand un logiciel veux envoyer des requêtes ARP, il doit s'enregistrer pour
un ethertype auprès du translator ARP, en lui précisant la taille des adresses
qu'il manipule, son adresse et un port de réponse. ARP va alors s'occuper tout
seul de répondre aux requêtes concernant l'ethertype enregistré, et va transférer
toutes les réponse qu'il reçoit via le port de réponse. De plus, en écrivant
sur le fichier d'ARP, il est possible de générer ses propres requêtes.

\subsection{Translator IP}
Le translator IP implémente l'IPv4 et ne supporte pour l'instant que TCP comme
protocole de niveau supérieure. Il s'occupe de la traduction des addresses IP
en addresses MAC en négociant avec le translator ARP et cache les résultats.

Il n'est cependant pas encore complètement testé et fonctionnel.

\subsection{Translator TCP}
Un début d'implémentation de TCP a été fait, mais rien de fonctionnel ni de
présentable cependant.

\section{Les difficultés}
Ce projet aurait été très simple si la seule documentation valable d'Hurd dont
nous disposions n'avait pas été son code source. La majorité du temps passé sur
ce projet n'a pas été occupé à développer le projet mais à lire le code source
de Hurd, GNU Mach et un peu la glibc pour Hurd, puis à faire des petits programmes
de test pour comprendre les APIs dont nous disposions. On peut les trouver sur
\href{https://github.com/lucas8/ipc-mach}{github}.

\section{Comment tester ?}
Le code est disponible sur \href{https://github.com/lucas8/ENS_sysres}{github}.
Cependant il doit être compilé sous Hurd, et installer Hurd est non trivial.
On a donc fait une \href{https://download.dwarfmaster.net/hurd.img.xz}{image
qemu} qui contient tout pour tester. Le mot de passe \emph{root} est
\texttt{root}, et les sources sont disponibles dans les dossier
\texttt{/home/user}. Un script \texttt{update.sh} y est présent qui recompile
tout et installe les translators.

Cependant, pour pouvoir recevoir des paquets, il faut configurer qemu en tun/tap.
Sur le remote qithub un script est disponible configurant le PC hôte (il a besoin
d'une version d'\texttt{ip} récente). Il suffit ensuite de lancer qemu de la façon
suivante : \texttt{qemu-kvm -m 1G -drive cache=writeback,file=hurd.img -net nic
-net tap,ifname=tap0,script=no,downscript=no}.

\end{document}

